\documentclass [12 pt ] {beamer}

\usepackage[T2A]{fontenc}
\usepackage[utf8]{inputenc}
\usepackage[english,russian]{babel}
\usepackage{amssymb,amsfonts,amsmath,mathtext}
\usepackage{cite,enumerate,float,indentfirst}
\usepackage[export]{adjustbox}
\usepackage{paratype} % шрифты
\usepackage{wrapfig}

\usepackage{listings}
\lstset{
    language=Haskell,%  Язык указать здесь
    basicstyle=\small,
    breaklines=true,%
    showstringspaces=false%
    inputencoding=utf8x%
}
\usetheme {metropolis}

\title{Генерация экземпляров классов типов \\ 
    на основе экземпляров производных классов в языке Haskell}
\author{\small О.\,Е.~Филиппская }
\institute{\textit{Направление подготовки:} Прикладная математика и информатика\\
\textit{Руководитель:} асс. каф. ИВЭ А.\,М.~Пеленицын\\

\vspace{8pt}

Южный федеральный университет\\
Институт математики, механики и компьютерных наук\\
имени~И.\,И.\,Воровича\\

\vspace{8pt}

Кафедра информатики и вычислительного эксперимента}
\date{ }

\begin {document}
    \maketitle
    \section{Постановка задачи}
    \begin {frame} {Постановка задачи}
     <Какой-то умный текст>
    \end {frame}
    \begin {frame}[containsverbatim]{GHC < 7.10}
     \begin{lstlisting}
class Monad m where  
    return :: a -> m a
    (>>=)  :: forall a b. m a 
                -> (a -> m b) -> m b
    (>>)   :: forall a b. m a 
                -> m b -> m b
    fail   :: String -> m a
     \end{lstlisting}
    \end {frame}
    
    \begin {frame}[containsverbatim]{GHC >= 7.10}
     \begin{lstlisting}
class Applicative m => Monad m where  
    return :: a -> m a
    (>>=)  :: forall a b. m a 
                -> (a -> m b) -> m b
    (>>)   :: forall a b. m a 
                -> m b -> m b
    fail   :: String -> m a
     \end{lstlisting}
    \end {frame}
\end {document}
